\documentclass[12pt]{article}%
\usepackage[T1]{fontenc}%
\usepackage[utf8]{inputenc}%
\usepackage{lmodern}%
\usepackage{textcomp}%
\usepackage{lastpage}%
%
\usepackage{amsmath}%
\usepackage{amssymb}%
\usepackage{geometry}%
\geometry{margin=1in}%
%
\begin{document}%
\normalsize%
\section{Cálculo de la inversa mediante cofactores y adjunta}%
\label{sec:Clculodelainversamediantecofactoresyadjunta}%
Se muestra el procedimiento paso a paso. Cada paso presenta la operación y la matriz resultante o la expresión matemática en modo display.%
\subsection{Paso 1: Matriz original}%
\label{subsec:Paso1Matrizoriginal}%
\[ A \;=\; \begin{bmatrix}2 & 1 & 3 \\ 1 & 0 & 2 \\ 3 & 4 & 1\end{bmatrix} \]

%
\subsection{Paso 2: Determinante}%
\label{subsec:Paso2Determinante}%
\[ \det(A) \;=\; 1 \]

%
\subsection{Paso 3: Menores y Cofactores (cada M\_\{ij\}, \textbackslash{}det(M\_\{ij\}) y C\_\{ij\})}%
\label{subsec:Paso3MenoresyCofactores(cadaMij,det(Mij)yCij)}%
\textbf{Menor } $M_{11}$
\[ \begin{bmatrix}0 & 2 \\ 4 & 1\end{bmatrix} \]
\[ \begin{aligned}\det(M_{11}) &= -8 \\[4pt]C_{11} &= (-1)^{1+1}\det(M_{11}) = -8\end{aligned} \]

\textbf{Menor } $M_{12}$
\[ \begin{bmatrix}1 & 2 \\ 3 & 1\end{bmatrix} \]
\[ \begin{aligned}\det(M_{12}) &= -5 \\[4pt]C_{12} &= (-1)^{1+2}\det(M_{12}) = 5\end{aligned} \]

\textbf{Menor } $M_{13}$
\[ \begin{bmatrix}1 & 0 \\ 3 & 4\end{bmatrix} \]
\[ \begin{aligned}\det(M_{13}) &= 4 \\[4pt]C_{13} &= (-1)^{1+3}\det(M_{13}) = 4\end{aligned} \]

\textbf{Menor } $M_{21}$
\[ \begin{bmatrix}1 & 3 \\ 4 & 1\end{bmatrix} \]
\[ \begin{aligned}\det(M_{21}) &= -11 \\[4pt]C_{21} &= (-1)^{2+1}\det(M_{21}) = 11\end{aligned} \]

\textbf{Menor } $M_{22}$
\[ \begin{bmatrix}2 & 3 \\ 3 & 1\end{bmatrix} \]
\[ \begin{aligned}\det(M_{22}) &= -7 \\[4pt]C_{22} &= (-1)^{2+2}\det(M_{22}) = -7\end{aligned} \]

\textbf{Menor } $M_{23}$
\[ \begin{bmatrix}2 & 1 \\ 3 & 4\end{bmatrix} \]
\[ \begin{aligned}\det(M_{23}) &= 5 \\[4pt]C_{23} &= (-1)^{2+3}\det(M_{23}) = -5\end{aligned} \]

\textbf{Menor } $M_{31}$
\[ \begin{bmatrix}1 & 3 \\ 0 & 2\end{bmatrix} \]
\[ \begin{aligned}\det(M_{31}) &= 2 \\[4pt]C_{31} &= (-1)^{3+1}\det(M_{31}) = 2\end{aligned} \]

\textbf{Menor } $M_{32}$
\[ \begin{bmatrix}2 & 3 \\ 1 & 2\end{bmatrix} \]
\[ \begin{aligned}\det(M_{32}) &= 1 \\[4pt]C_{32} &= (-1)^{3+2}\det(M_{32}) = -1\end{aligned} \]

\textbf{Menor } $M_{33}$
\[ \begin{bmatrix}2 & 1 \\ 1 & 0\end{bmatrix} \]
\[ \begin{aligned}\det(M_{33}) &= -1 \\[4pt]C_{33} &= (-1)^{3+3}\det(M_{33}) = -1\end{aligned} \]

%
\subsection{Paso 4: Matriz de cofactores C}%
\label{subsec:Paso4MatrizdecofactoresC}%
\[ C \;=\; \begin{bmatrix}-8 & 5 & 4 \\ 11 & -7 & -5 \\ 2 & -1 & -1\end{bmatrix} \]

%
\subsection{Paso 5: Adjunta (adjugate)}%
\label{subsec:Paso5Adjunta(adjugate)}%
\[ \operatorname{adj}(A) \;=\; C^{T} \;=\; \begin{bmatrix}-8 & 11 & 2 \\ 5 & -7 & -1 \\ 4 & -5 & -1\end{bmatrix} \]

%
\subsection{Paso 6: Fórmula de la inversa}%
\label{subsec:Paso6Frmuladelainversa}%
\[ A^{-1} \;=\; \frac{1}{\det(A)}\,\operatorname{adj}(A) \;=\; \frac{1}{1}\,\begin{bmatrix}-8 & 11 & 2 \\ 5 & -7 & -1 \\ 4 & -5 & -1\end{bmatrix} \]

%
\subsection{Paso 7: Matriz inversa (entradas explícitas)}%
\label{subsec:Paso7Matrizinversa(entradasexplcitas)}%
\[ A^{-1} \;=\; \begin{bmatrix}-8 & 11 & 2 \\ 5 & -7 & -1 \\ 4 & -5 & -1\end{bmatrix} \]

%
\end{document}